\documentclass[10pt]{beamer}
\usepackage{minted}
\usepackage{graphicx}
\usetheme{metropolis}
\usepackage{appendixnumberbeamer}

\usepackage{booktabs} \usepackage[scale=2]{ccicons}

\usepackage{pgfplots}
\usepgfplotslibrary{dateplot}

\usepackage{xspace}
\newcommand{\themename}{\textbf{\textsc{metropolis}}\xspace}
\usepackage{fontspec}

\setmonofont{Fira Mono}

\title{LISP}
\subtitle{A Programmable Programming Language}
\date{25.01.2017}
\author{Christoph Müller}
\institute{Offenes Kolloquium für Informatik}
% \titlegraphic{\hfill\includegraphics[height=1.5cm]{logo.pdf}}

\begin{document}

\maketitle

\begin{frame}{Agenda}
  \setbeamertemplate{section in toc}[sections numbered]
  \tableofcontents[hideallsubsections]
\end{frame}

\section{Why Lisp?}

\begin{frame}{Timeless}
    \begin{itemize}
        \item When we talk about Lisp we talk about a language family
        \item One of the oldest ($\sim 1958$) language families still in use today (only Fortran is older)
        \item The Syntax is by its very nature timeless
    \end{itemize}
    \centering{\includegraphics[width=0.7\textwidth]{img/history.png}}
\end{frame}



% LISP. HigherOrderFunctions, Functors, LexicalClosures, GarbageCollection, amazingly powerful macros (see LispMacros), MultipleDispatch, MetaObjectProtocol, Introspection and Reflection, Dynamic Objects, MethodCombinations, MixIns, Code as Data, IncrementalCompilation, DoWhatImean, Continuations (see ContinuationExplanation), ExpertSystem Shells, Frames, LotsofIrritatingSillyParentheses, if-then-else statements (previous languages, such as Fortran, only had a conditional goto). Note that LispLanguage predates CobolLanguage and AlgolLanguage. McCarthy's original 1958 LISP interpreter had the if-then-else statement, and was the first language to implement it, as FortranLanguage of the time was quite primitive.
\begin{frame}{Innovator}
    \begin{itemize}
        \item Garbage Collection
        \item Homoiconicity (Code is Data)
        \item Higher Order Functions
        \item Dynamic Typing
        \item Read Evaluate Print Loop (REPL)
        \item Multiple Dispatch
        \item And many more ...
    \end{itemize}
\end{frame}

\begin{frame}{Scheme - A Language for Teaching}
    \begin{itemize}
        \item Scheme was used as an introductory Language in famous Universities like MIT (6.001)
        \item Extremely small language core
        \item Great for learning to build your own abstractions 
    \end{itemize}
    \centering{\includegraphics[width=0.3\textwidth]{img/scheme.jpeg}}
\end{frame}

\begin{frame}{Picking a Language for this Talk}
    Lets look at the most popular Lisp dialects on GitHub (provided by GitHut):
    \begin{center}
        \begin{tabular}{ | l | l | }
            \hline
            \textbf{GitHub Popuplarity Rank} & \textbf{Language} \\ \hline
            20 & Emacs Lisp \\ \hline
            23 & \textbf{Clojure} \\ \hline
            40 & \textbf{Scheme} \\ \hline
            42 & Common Lisp \\ \hline
            48 & Racket \\
            \hline
        \end{tabular}
    \end{center}

    Clojure with its JVM heritage and Scheme with its focus on a small core will be used throughout this talk.
\end{frame}

\section{The Basics of Lisp}

\begin{frame}[fragile]{The name giving lists}
    \begin{itemize}
        \item The basis of lisp is the s(ymbolic)-expression
        \item Either a atom or a list
        \item Atoms are either symbols or literals
        \item Every element of a list is either an atom or another list
        \item Elements are separated by whitespace
        \item The first element of a (to be evaluated) list has to be what we will call a \emph{verb} in this talk
    \end{itemize}
    \begin{minted}{clojure}
        ;atoms
        x
        12
        ;lists
        (+ 1 2 3)
        (+ (* 2 3) 3)
    \end{minted}
\end{frame}

\begin{frame}[fragile]{What is a verb?}
    \begin{itemize}
        \item A verb is either a
        \begin{itemize}
            \item A function
            \item A macro
            \item A special form
        \end{itemize}
        \item Special forms include \emph{if, fn, loop, recur etc.}
        \item They are built into the language and cannot be user defined
        \item On the other hand functions and macros can be
        \item Since functions are familiar to most people we will start with them
    \end{itemize}
\end{frame}

\begin{frame}[fragile]{Calling Functions}
    \begin{itemize}
            \item The arguments of functions are evaluated before they are passed to the function
            \item This is an important distinction from macros/special forms
            \item Calling functions in a prefix manner might feel strange in the beginning 
    \end{itemize}
    \begin{minted}{clojure}
        ;the + function called as a 
        ;prefix and not as an infix
        (+ 1 2 3)
        ;the infamous
        (println "hello world")
    \end{minted}
\end{frame}

\begin{frame}[fragile]{Calling Java Methods -- Clojure Only}
    \begin{itemize}
        \item Since Clojure runs on the JVM, interop with Java is necessary to make use of existing libraries
        \item Java Methods are called like (.instanceMember instance args*)
    \end{itemize}
    \begin{minted}{clojure}
        (.toUpperCase "Hello World")
        -> "HELLO WORLD"
    \end{minted}
    \begin{itemize}
        \item Creating a new Instance will be very familiar to Java Developers
        \item There is however a short form for creating new instances
    \end{itemize}
    \begin{minted}{clojure}
        (new String "hello world")
        -> "hello world"
        (String. "hello world")
        -> "hello world"
    \end{minted}
\end{frame}

\begin{frame}[fragile]{Just a bit more Syntax}
    \begin{itemize}
            \item Before we will learn how to create our own functions a bit more syntastic sugar
            \item Vectors are the data structure in Clojure that are used to define the arguments of a function
    \end{itemize}
    \begin{minted}{clojure}
    [1 2 3]
    -> [1 2 3]
    (vector 1 2 3)
    -> [1 2 3]
    \end{minted}
    \begin{itemize}
            \item Maps/Dictionaries are created via the curly brace literal
    \end{itemize}
    \begin{minted}{clojure}
    {"a" 1 "b" 2 "c" 3}; or (hash-map ...)
    -> {"a" 1, "b" 2, "c" 3}
    ; note the comma, comma is whitespace in Clojure
    \end{minted}
    \begin{itemize}
            \item These are implemented via so called reader macros we will learn about them in the macro section
    \end{itemize}
\end{frame}

\begin{frame}[fragile]{Define your own Functions - 1}
    \begin{itemize}
        \item The special form \emph{fn} is used to create functions
    \end{itemize}
    \begin{minted}{clojure}
    (fn [x] (* x x))
    -> #function[user/eval10725/fn--10726]
    ((fn [x] (* x x)) 12)
    -> 144
    \end{minted}
    \begin{itemize}
        \item An optional name can be given to the function to make non tail calls
    \end{itemize}
    \begin{minted}{clojure}
    ((fn foo [x] (if (< x 1) x (foo (dec x)))) 10)
    -> 0
    \end{minted}
\end{frame}

\begin{frame}[fragile]{Define your own Functions - 2}
    \begin{itemize}
        \item to make a tail recursive call the \emph{recur} special form is used
    \end{itemize}
    \begin{minted}{clojure}
    ((fn [x] (if (< x 1) x (recur (dec x)))) 10)
    \end{minted}
    \begin{itemize}
        \item Since functions will often be bound to a global variable (inside a namespace) the following syntax will be seen often
    \end{itemize}
    \begin{minted}{clojure}
    (defn foo "doc string here" [x]
      (if (< x 1) 
        x 
        (foo (dec x)))) 
    -> #'user/foo
    (foo 10)
    -> 0
    \end{minted}
\end{frame}

\begin{frame}[fragile]{Define your own Functions - 3}
    \begin{itemize}
        \item For short lambda functions there is an even more compact notation
        \item inside the lambda function \% is used to for arguments
        \item \% and \%1 are used for the first argument, \%2 ... for the rest
    \end{itemize}
    \begin{minted}{clojure}
    #(* % %)
    -> #function[user/eval10725/fn--10726]
    (map #(* % %) (range 10))
    -> (0 1 4 9 16 25 36 49 64 81)
    \end{minted}
\end{frame}

\begin{frame}[fragile]{Branch with \emph{if}}
    \begin{itemize}
        \item We have already seen the \emph{if} special form
        \item It consists of a test, a then expression and an optional else expression
        \item \emph{if} can be used like a ternary expression in Java
    \end{itemize}
    \begin{minted}{clojure}
    (println (if (< 4 3) "hello" "world"))
    -> world
    \end{minted}
    \begin{minted}{Java}
    System.out.println(4 < 3 ? "hello" : "world")
    \end{minted}
\end{frame}

\begin{frame}[fragile]{\emph{do} multiple things}
    \begin{itemize}
        \item Evaluates multiple expressions and returns the value of the last one (or nil)
    \end{itemize}
    \begin{minted}{clojure}
    (if (< 3 4) 
        (do 
            (println "hello world") 
            (println "and again")))
    -> hello world
       and again
    \end{minted}
\end{frame}

\begin{frame}[fragile]{Bind with let}
    \begin{itemize}
            \item Of course we also need to bind local variables inside expressions
            \item The \emph{let} special form is used for that
            \item It uses pairs inside a vector for that purpose
            \item Has support for Destructering
    \end{itemize}
    \begin{minted}{clojure}
    (let [x 1] x)
    -> 1
    ;basic Destructering
    (let [[x y] [1 2]] (+ x y))
    -> 3
    \end{minted}
\end{frame}

\begin{frame}[fragile]{Loop with ... well ... loop}
    \begin{itemize}
        \item We have seen recursion, now we will cover iteration with the \emph{loop} special form
        \item The \emph{loop} form is very similiar to a \emph{let} binding
        \item To repeat we use \emph{recur} just like when working with tail recursion earlier
    \end{itemize}
    \begin{minted}{clojure}
    (loop [x 10] 
      (if (> x 1)
        (recur (- x 2))))
    \end{minted}
    \begin{itemize}
        \item There are other types of loops in clojure, like \emph{for} and \emph{while}, but they are implemented as macros
        \item \emph{loop} and \emph{recur} is therefore all we need!
    \end{itemize}
\end{frame}

\begin{frame}[fragile]{Your new best friends \emph{doc} and \emph{source}}
    \begin{itemize}
        \item \emph{doc} will show you the docstring of a given function, macro or special form
    \end{itemize}
    \begin{minted}{clojure}
    (doc +)
    -> ([] [x] [x y] [x y & more])
       Returns the sum of nums. (+) returns 0. 
       Does not auto-promote longs,
       will throw on overflow. See also +'
    \end{minted}
    \begin{itemize}
        \item \emph{source} will show you the source code of a given function or macro
    \end{itemize}
    \begin{minted}{clojure}
    (source when)
    -> (defmacro when
         ".. doc string ..."
         [test & body]
         (list 'if test (cons 'do body)))
    \end{minted}
\end{frame}

\section{Macros in Action}

\begin{frame}{Kinds of Macros}
    Macros can be grouped in different Categories
    \begin{itemize}
        \item Syntactic Sugar Macros - Using simple pattern matching and templates
        \item Complex Transformations - The most demanding and the most rewarding
        \item Reader Macros - Syntactic sugar on the reader level, not to be confused with the other two
    \end{itemize}
\end{frame}

\begin{frame}[fragile]{Yes Code really is Data}
    \begin{itemize}
        \item Code really is nothing more than a linked list
    \end{itemize}
    \begin{minted}{clojure}
    (type '(+ 1 2 3))
    -> clojure.lang.PersistentList
    \end{minted}
    \begin{itemize}
        \item The ' is used to \textbf{prevent} evaluation, it is equivalent to \emph{(quote ...)}
        \item The function \emph{eval} (may be familiar from a lot of scripting languages) takes a s-expression, not a string!
    \end{itemize}
    \begin{minted}{clojure}
    (eval '(+ 1 2 3))
    -> 6
    \end{minted}
\end{frame}

\begin{frame}[fragile]{Reader Macros}
    \begin{itemize}
        \item To get s-expression from a string, the \emph{read-string} function can be used
    \end{itemize}
    \begin{minted}{clojure}
    (eval (read-string "(+ 1 2 3)"))
    -> 6
    \end{minted}
    \begin{itemize}
        \item The reader uses read macros to parse special syntax like [], the ' or the lambda \#() syntax
        \item Clojure has a set of predefined reader macros, they can not be user defined
        \item Some lisps (e.g. Common Lisp or Racket) don't suffer from this restriction
        \item That means that theses lisp dialects have compile time + parse time macros 
    \end{itemize}
\end{frame}

\begin{frame}[fragile]{The syntax quote}
    \begin{itemize}
        \item Before we will look at macros, we will introduce the syntax-quote
        \item It helps us evaluating things in nested quoted structeres
    \end{itemize}
    \begin{minted}{clojure}
    ; ` is a syntax quote
    ; ~ evaluates inside such quote
    `(foo ~(+ 1 2 3))
    -> (user/foo 6)
    \end{minted}
    \begin{itemize}
        \item Everything that is not evaluated through the tilde character will be left alone (only the current namespace is added)
    \end{itemize}
\end{frame}

\begin{frame}[fragile]{Macros}
    \begin{itemize}
        \item Macros are arbitrary lisp code executed at compile time
        \item Normally only code transformation is done, but it is not limited to transformations
        \item One could for instance query a database or perform computation of all sorts
    \end{itemize}
    \begin{minted}{clojure}
    (defmacro foo [] (+ 1 2))
    -> 3
    \end{minted}
\end{frame}

\begin{frame}[fragile]{A bit of sugar}
    \begin{itemize}
        \item Let's look at the simple \emph{when} macro with \emph{(source when)}
    \end{itemize}
    \begin{minted}{clojure}
    (defmacro when
      ...
      [test & body]
      (list 'if test (cons 'do body)))
    \end{minted}
    \begin{itemize}
        \item The ampersand just stands for the rest, so body are all expressions after test.
        \item Here we create code by creating a list 
        \item Using the list function is basically the inverse of a syntax quote, everything is evaluated except quoted expressions
    \end{itemize}
\end{frame}

\begin{frame}[fragile]{The \emph{while} loop}
    \begin{itemize}
        \item The \emph{while} loop ships with Clojure, here is the source
    \end{itemize}
    \begin{minted}{clojure}
    (defmacro while
      ...
      [test & body]
      `(loop []
         (when ~test
           ~@body
           (recur))))
    \end{minted}
    \begin{itemize}
        \item The loop does not need bindings, since were ware dealing with a while and not a for loop
        \item The @-sign in front of body unpacks the body list into its elements 
        \item So 1 2 3 4 \dots instead of (1 2 3 4 \dots)
    \end{itemize}
\end{frame}

\begin{frame}[fragile]{Pattern Matching in Scheme}
    \begin{itemize}
        \item Scheme provides an even more elegant syntax for simple macros with \emph{syntax-rules}
    \end{itemize}
    \begin{minted}{scheme}
    (define-syntax for
      (syntax-rules (in as)
         (
          ;pattern
          (for element in list body ...)
          ;template
          (map (lambda (element) body ...) list)
         )
         ((for list as element body ...)
          (for element in list body ...))))

    (for i in '(1 2 3) (display i))
    \end{minted}
\end{frame}

\begin{frame}[fragile]{Complex Macros}
    \begin{itemize}
        \item For time reasons we can't look at more complex macros in detail
        \item We will look at an example at a higher level
        \item As an example I have picked the async/await statement from languages like C\# or JS (ECMAScript 2017 draft)
        \item This is usally done by creating a state machine
        \item The CSharp code was decompiled to retrieve the state machine
    \end{itemize}
\end{frame}

\begin{frame}[fragile]{Async in C\# - Before}
    \inputminted[fontsize=\footnotesize]{csharp}{csharp-before.cs}
\end{frame}

\begin{frame}[fragile]{Async in C\# - After}
    The CLR has no extra byte code instructions for async/await, everything is handled by the compiler 
    \inputminted[fontsize=\footnotesize]{csharp}{csharp-after.cs}
\end{frame}

\begin{frame}[fragile]{Async in Clojure with core.async}
    \begin{itemize}
        \item In Clojure we don't need dont to wait until a standard comittee adds the feature
        \item We just use the core.async library implemented purely in Clojure
        \item core.async uses channels in a technique known as Communicating Sequential Processes
        \item The syntax will be more familiar to users of the go language
    \end{itemize}
    \begin{minted}{clojure}
    (defn what-is-the-answer [c]
      (go
        ;timeout creates a new channel
        (<! (timeout 2.3652E17))
        (>! c 42)))
    \end{minted}
\end{frame}

\begin{frame}[fragile]{Async in Clojure with core.async}
    The \emph{macroexpand} function helps us to examine the macro code
    \begin{minted}[fontsize=\footnotesize]{clojure}
(fn state-machine
  ([state_3730]
    (loop []
      (let
        [result 
          (case (int (ioc/aget-object state_3730 1))
           3 (let [inst_3728 (ioc/aget-object state_3730 2)
                   state_3730 state_3730]
               (ioc/return-chan state_3730 inst_3728))
           2 (let [inst_3725 (ioc/aget-object state_3730 2)
                   inst_3726 (vector kind query)
                   state_3730 (ioc/aset-all! state_3730 5 inst_3725)]
               (ioc/put! state_3730 3 c inst_3726))
           1 (let [inst_3722 (rand-int 100)
                   inst_3723 (timeout inst_3722)
                   state_3730 state_3730]
               (ioc/take! state_3730 2 inst_3723)))]
    \end{minted}
\end{frame}

%(defmacro unless [pred then else] `(if ~pred ~else ~then))

\begin{frame}{Deep Understanding}
    Clojure provides a deep understanding of the language through macros and functions like \emph{doc}, \emph{source} and \emph{macroexpand}. This should not be taken for granted, especially when compared to languages like e.g. C++.
\end{frame}

\begin{frame}{C++ - The worst Offender}
    \begin{quote}
        Help me sort out the meaning of "\{\}" as a constructor argument
    \end{quote}
    \hspace*\fill{-- Scott Meyers, Author of \emph{Effective C++}}
    \inputminted[fontsize=\footnotesize]{cpp}{scott.cpp}
\end{frame}

\begin{frame}{C++ - The worst Offender}
    \begin{itemize}
            \item The specific example can be looked up on Scott Meyers Blog
            \item The last call does \emph{not} create an empty list
            \item Even a seasoned C++ expert and book author can't figure out seemingly simple examples
            \item -> Please don't take the tools Clojure provides for granted
    \end{itemize}
\end{frame}

\begin{frame}{Be carefull with Macros}
    \begin{itemize}
            \item Use them only when a function won't do
            \item Macros tend to ``creep'' up the call chain
            \item Writing good Macros takes quite a bit of practice
            \item Good Error messages (hello Rust) are now your responsibility
            \item Since you are now the ``compiler engineer''
    \end{itemize}
\end{frame}

\section{Tools and Platforms}

\begin{frame}{Emacs}
    \begin{itemize}
            \item The most important IDE for Lisp is of course emacs
            \item Written in EmacsLisp and with support for pretty much every Lisp dialect
            \item For Clojure the cider package is recommended
    \end{itemize}
\end{frame}

\begin{frame}{Emacs}
    \centering{\includegraphics[width=\textwidth]{img/cider.png}}
\end{frame}

\begin{frame}{Clojure IDE Support}
    \begin{itemize}
            \item For Clojure the two famous IDEs Eclipse and IntelliJ both have support via plugins
            \item Eclipse has the Counter Clockwise Wise Plugin
            \item IntelliJ has the (commercial) Cursive IDE
    \end{itemize}
\end{frame}

\begin{frame}{Eclipse}
    \centering{\includegraphics[width=\textwidth]{img/ccw.png}}
\end{frame}

\begin{frame}{IntelliJ}
    \centering{\includegraphics[width=\textwidth]{img/cursive.png}}
\end{frame}

\begin{frame}{Vim}
    \begin{itemize}
            \item It is of course heresy to use Vim for Lisp, but I do it anyway 
            \item Tim Pope has written the great fireplace.vim plugin
            \item Will not be considered a full blown IDE by most people
    \end{itemize}
\end{frame}

\begin{frame}{Tools}
    \begin{itemize}
            \item Most Lisps have tools for project and package/dependency management
            \item For Clojure there is leiningen and boot, with leiningen being the older and more popular tool
            \item The key difference is the declarative approach with leiningen, while boot just uses plain clojure logic
            \item For the web developers there is an analogy: leiningen -> grunt, boot -> gulp
    \end{itemize}
\end{frame}

\section{Literature and more obscure Lisp dialects}


\begin{frame}{Books on Clojure}
    The left one is available for free from \url{clojure-buch.de}
    \vspace{0.5cm}
\begin{columns}[t]
\column{.5\textwidth}
\centering
\includegraphics[width=3.5cm]{./img/buch.jpg}
\column{.5\textwidth}
\centering
\includegraphics[width=3.5cm]{./img/master.jpg}
\end{columns}
\end{frame}

\begin{frame}{Shen Language}
    \begin{itemize}
        \item If clojure and scheme are not enough for you, take a look at \url{shenlanguage.org}
        \item We will briefly look at a few features 
        \item Static type checking based on the sequent calculus
        \item Integrated fully functional Prolog (defprolog \dots)
        \item Inbuilt compiler-compiler (Shen-YACC) based on BNF notation
        \item Can be used to develop efficient compilers for programming languages and transducers for natural language processing
        \item \dots The ``Everything but the kitchen sink''-Lisp
    \end{itemize}
\end{frame}

\section{Why it never (really) caught on}

\begin{frame}{Performance}
    \begin{quote}
        A Lisp programmer knows the value of everything, but the cost of nothing.
    \end{quote}
    \hspace*\fill{-- Alan Perlis, American Computer Scientist}
    \begin{itemize}
        \item Historically a lot of Lisps features required quite a bit of processing power or special hardware support (see Lisp machines)
        \item Today that argument is mostly moot 
        \item The performance of e.g. Clojure is in the same ballpark as most dynamic JVM languages
    \end{itemize}
\end{frame}


\begin{frame}{Syntax}
    \begin{quote}
One of the ideas I keep stressing in the design of Perl is that things that ARE different should LOOK different. The reason many people hate programming in Lisp is because everything looks the same. I've said it before, and I'll say it again: Lisp has all the visual appeal of oatmeal with fingernail clippings mixed in. (Other than that, it's quite a nice language.)
    \end{quote}
    \hspace*\fill{-- Larry Wall, Creator of the Perl Language}
\end{frame}

\begin{frame}{Syntax}
    \begin{itemize}
        \item Maybe a gate keeper for language features (a committee or a ``benevolent dictator'') is exactly what you want
        \item In real businesses it is important that existing code can be understood and altered quickly and cost effectively
        \item Evolving languages with fixed feature sets may help in the hiring process 
    \end{itemize}
\end{frame}

\begin{frame}{Tooling}
    \begin{itemize}
            \item A lot of Lisp dialects (especially Scheme implementations) tend to isolate themselves from other eco systems
            \item Most people are not interested in calling C libraries using some FFI
            \item That only leaves the existing libraries written in that particular dialect
            \item They have at best emacs support, which just does not cut it anymore in 2017 for most developers
            \item Refactoring, Linting and Formatting tools (like in languages like go) are often missing
    \end{itemize}
\end{frame}

\begin{frame}{Community}
    \begin{itemize}
            \item The so called ``Smug Lisp Weenie'':
                \begin{quote}
                    Someone who is always eager to let you know that whatever it is you're doing would be done better in Lisp.
                \end{quote}
            \item Lets just say they have ``strong opinions''
    \end{itemize}
\end{frame}

\begin{frame}[standout]
    Thank You!
\end{frame}

\section{Bonus - A bit of History}

\end{document}
